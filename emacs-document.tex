% Created 2015-10-20 Tue 08:12
\documentclass[10pt,a4paper]{article}
\usepackage{graphicx}
\usepackage{xcolor}

\usepackage{lmodern}
\usepackage{verbatim}
\usepackage{fixltx2e}
\usepackage{longtable}
\usepackage{float}
\usepackage{tikz}
\usepackage{wrapfig}
\usepackage{soul}
\usepackage{textcomp}
\usepackage{listings}
\usepackage{geometry}
\usepackage{algorithm}
\usepackage{algorithmic}
\usepackage{marvosym}
\usepackage{wasysym}
\usepackage{latexsym}
\usepackage{natbib}
\usepackage{fancyhdr}
\usepackage{comment}

\usepackage{zhfontcfg} % added my own sty file under /usr/local/texlive/texmf-local/tex/latex/local

\usepackage[xetex,colorlinks=true,CJKbookmarks=true,
linkcolor=blue,
urlcolor=blue,
menucolor=blue]{hyperref}
\usepackage{xunicode,xltxtra}

\hypersetup{unicode=true}
\geometry{a4paper, textwidth=6.5in, textheight=10in,
marginparsep=7pt, marginparwidth=.6in}

\XeTeXlinebreakskip = 0pt plus 1pt

\definecolor{foreground}{RGB}{220,220,204}%浅灰
\definecolor{background}{RGB}{62,62,62}%浅黑
\definecolor{preprocess}{RGB}{250,187,249}%浅紫
\definecolor{var}{RGB}{239,224,174}%浅肉色
\definecolor{string}{RGB}{154,150,230}%浅紫色
\definecolor{type}{RGB}{225,225,116}%浅黄
\definecolor{function}{RGB}{140,206,211}%浅天蓝
\definecolor{keyword}{RGB}{239,224,174}%浅肉色
\definecolor{comment}{RGB}{180,98,4}%深褐色
\definecolor{doc}{RGB}{175,215,175}%浅铅绿
\definecolor{comdil}{RGB}{111,128,111}%深灰
\definecolor{constant}{RGB}{220,162,170}%粉红
\definecolor{buildin}{RGB}{127,159,127}%深铅绿
\punctstyle{kaiming}
\title{}
\fancyfoot[C]{\bfseries\thepage}
\chead{\MakeUppercase\sectionmark}
\pagestyle{fancy}
\tolerance=1000


\usepackage{xeCJK}
\setCJKmainfont{SimSun}
\author{Lu Jianmei\thanks{lu.jianmei@trs.com.cn}}
\date{\today}
\title{Emacs Document}
\hypersetup{
  pdfkeywords={org-mode Emacs jquery jquery.mobile jquery.ui wcm},
  pdfsubject={A notes that include all works and study things in 2015},
  pdfcreator={Emacs 24.5.1 (Org mode 8.2.10)}}
\begin{document}

\maketitle
\setcounter{tocdepth}{3}
\tableofcontents

\#+DOCUMENT 其它文档

\begin{itemize}
\item \href{../work-note-in-2015.org}{\{Back to Index\}}
\end{itemize}

\section{org-mode configuration}
\label{sec-1}
\subsection{Status config}
\label{sec-1-1}
Status详细配置解决说明参见:\url{http://orgmode.org/manual/In_002dbuffer-settings.html#In_002dbuffer-settings}
\lstset{basicstyle=\color{foreground}\small\mono,keywordstyle=\color{function}\bfseries\small\mono,identifierstyle=\color{doc}\small\mono,commentstyle=\color{comment}\small\itshape,stringstyle=\color{string}\small,showstringspaces=false,numbers=left,numberstyle=\color{preprocess},stepnumber=1,backgroundcolor=\color{background},tabsize=4,captionpos=t,breaklines=true,breakatwhitespace=true,showspaces=false,columns=flexible,frame=single,frameround=tttt,framesep=0pt,framerule=8pt,rulecolor=\color{background},fillcolor=\color{white},rulesepcolor=\color{comdil},framexleftmargin=10mm,language=LISP,label= ,caption= }
\begin{lstlisting}
#+STARTUP: overview
#+STARTUP: hidestars
#+STARTUP: logdone
#+PROPERTY: Effort_ALL  0:10 0:20 0:30 1:00 2:00 4:00 6:00 8:00
#+COLUMNS: %38ITEM(Details) %TAGS(Context) %7TODO(To Do) %5Effort(Time){:} %6CLOCKSUM{Total}
#+PROPERTY: Effort_ALL 0 0:10 0:20 0:30 1:00 2:00 3:00 4:00 8:00
#+TAGS: { OFFICE(o) HOME(h) } COMPUTER(c) PROJECT(p) READING(r) PROBLEM(b)
#+TAGS:  LUNCHTIME(l) ENGLISH(e)
#+SEQ_TODO: TODO(t) STARTED(s) WAITING(w) APPT(a) | DONE(d) CANCELLED(c) DEFERRED(f)
\end{lstlisting}
\subsection{Document Head config}
\label{sec-1-2}
Head详细的配置解释说明参见:\url{http://orgmode.org/manual/Export-settings.html}
\lstset{basicstyle=\color{foreground}\small\mono,keywordstyle=\color{function}\bfseries\small\mono,identifierstyle=\color{doc}\small\mono,commentstyle=\color{comment}\small\itshape,stringstyle=\color{string}\small,showstringspaces=false,numbers=left,numberstyle=\color{preprocess},stepnumber=1,backgroundcolor=\color{background},tabsize=4,captionpos=t,breaklines=true,breakatwhitespace=true,showspaces=false,columns=flexible,frame=single,frameround=tttt,framesep=0pt,framerule=8pt,rulecolor=\color{background},fillcolor=\color{white},rulesepcolor=\color{comdil},framexleftmargin=10mm,language=LISP,label= ,caption= }
\begin{lstlisting}
#+TITLE: Zhoushan Dangan Project Schedule
#+LANGUAGE:  zh
#+AUTHOR: Lu Jianmei
#+EMAIL: lu.jianmei@trs.com.cn
#+OPTIONS:   H:3 num:t   toc:3 \n:nil @:t ::t |:t ^:nil -:t f:t *:t <:t p:t pri:t
#+OPTIONS:   TeX:t LaTeX:nil skip:nil d:nil todo:t pri:nil tags:not-in-toc
#+OPTIONS:   author:t creator:t timestamp:t email:t
#+DESCRIPTION: A notes that include all works and study things in 2015
#+KEYWORDS:  org-mode Emacs jquery jquery.mobile jquery.ui wcm
#+INFOJS_OPT: view:nil toc:t ltoc:t mouse:underline buttons:0 path:http://orgmode.org/org-info.js
#+EXPORT_SELECT_TAGS: export
#+EXPORT_EXCLUDE_TAGS: noexport
#+LATEX_HEADER: \usepackage{xeCJK}
#+LATEX_HEADER: \setCJKmainfont{SimSun}
#+LATEX_CLASS: cn-article
#+STARTUP: logredeadline, logreschedule
#+ATTR_HTML: :border 2 :rules all :frame all
\end{lstlisting}
如果配置项目管理,可以按添加如下配置
\lstset{basicstyle=\color{foreground}\small\mono,keywordstyle=\color{function}\bfseries\small\mono,identifierstyle=\color{doc}\small\mono,commentstyle=\color{comment}\small\itshape,stringstyle=\color{string}\small,showstringspaces=false,numbers=left,numberstyle=\color{preprocess},stepnumber=1,backgroundcolor=\color{background},tabsize=4,captionpos=t,breaklines=true,breakatwhitespace=true,showspaces=false,columns=flexible,frame=single,frameround=tttt,framesep=0pt,framerule=8pt,rulecolor=\color{background},fillcolor=\color{white},rulesepcolor=\color{comdil},framexleftmargin=10mm,language=LISP,label= ,caption= }
\begin{lstlisting}
#+COLUMNS: #+COLUMNS: %25ITEM %DEADLINE %SCHEDULED %11Status %9Approved(Approved?){X} %TAGS %PRIORITY %TODO
#+Status_ALL: "In progress" "Not started yet" "Finished" ""
#+Approved_ALL: "[ ]" "[X]"
\end{lstlisting}

\subsection{Table and Image}
\label{sec-1-3}
表格导出与图片导出成latex时,需要设置相应的导出latex参数,具体参数参见:\url{http://orgmode.org/org.html#LaTeX-specific-attributes}
\lstset{basicstyle=\color{foreground}\small\mono,keywordstyle=\color{function}\bfseries\small\mono,identifierstyle=\color{doc}\small\mono,commentstyle=\color{comment}\small\itshape,stringstyle=\color{string}\small,showstringspaces=false,numbers=left,numberstyle=\color{preprocess},stepnumber=1,backgroundcolor=\color{background},tabsize=4,captionpos=t,breaklines=true,breakatwhitespace=true,showspaces=false,columns=flexible,frame=single,frameround=tttt,framesep=0pt,framerule=8pt,rulecolor=\color{background},fillcolor=\color{white},rulesepcolor=\color{comdil},framexleftmargin=10mm,language=LISP,label= ,caption= }
\begin{lstlisting}
#+ATTR_LATEX: :environment longtable :align l|lp{3cm}r|l
#+ATTR_LATEX: :mode math :environment bmatrix :math-suffix \times
\end{lstlisting}
\lstset{basicstyle=\color{foreground}\small\mono,keywordstyle=\color{function}\bfseries\small\mono,identifierstyle=\color{doc}\small\mono,commentstyle=\color{comment}\small\itshape,stringstyle=\color{string}\small,showstringspaces=false,numbers=left,numberstyle=\color{preprocess},stepnumber=1,backgroundcolor=\color{background},tabsize=4,captionpos=t,breaklines=true,breakatwhitespace=true,showspaces=false,columns=flexible,frame=single,frameround=tttt,framesep=0pt,framerule=8pt,rulecolor=\color{background},fillcolor=\color{white},rulesepcolor=\color{comdil},framexleftmargin=10mm,language=LISP,label= ,caption= }
\begin{lstlisting}
;;图片
#+ATTR_LATEX: :width 5cm :options angle=90
[[./img/sed-hr4049.pdf]]
#+ATTR_LATEX: :caption \bicaption{HeadingA}{HeadingB}
[[./img/sed-hr4049.pdf]]
\end{lstlisting}
\section{Manual}
\label{sec-2}
\subsection{Checking for manual}
\label{sec-2-1}
通过emacs可以查看elisp的接口文档,从中间找到各个功能模块的变量定义等;
操作办法: \textbf{M-: (info "(elisp) Memory Usage")}
\subsection{Org Mode}
\label{sec-2-2}
\subsubsection{Property}
\label{sec-2-2-1}
属性可以用于显示在column视图中,或者agenda view中,增加对任务的描述,描述可以是状态、属性补充等目标;它具有属性的继承性,因此在项目根结点上配置属性后,即可以在子任务中使用这些属性的备选值,使用快捷键进行切换;
\begin{itemize}
\item 配置根属性的备选值
在项目根节点上,配置如下属性,即可以通过 \textbf{C-c C-x p} 命令添加一条属性,并将光标定位在属性上后,通过 \textbf{S-left/right} 切换所需的属性值;
\end{itemize}
\subsubsection{Column view}
\label{sec-2-2-2}
Column view是建立于org-mode任务管理之上的快速以表格查看各个任务情况的视图,可以使用 \textbf{C-c C-x C-c} 来打开column view,而此种查看方式只是一种查看,并不会被打印,如果需要通过动态管理column view,并支持打印,可以通过 \textbf{C-c C-x i} 插入一个动态的column部分在文件中,但此部分不是动态关联的,即修改了任务内容,插入的column view是不会更新的,但可以通过执行 \textbf{C-c C-x C-u} 进行更新;注:已经通过配置(add-hook 'before-save-hook 'org-update-all-dblocks)达到在保存文件时,即会更新文件中所有的column view中的数据;
\subsubsection{Capture}
\label{sec-2-2-3}
Capture用于快速记录任务或者idea内容,并支持在后面进行capture的内容refile;
\subsubsection{Agenda View}
\label{sec-2-2-4}
\begin{enumerate}
\item Agenda基本命令使用
\label{sec-2-2-4-1}
\begin{enumerate}
\item Agenda View用于以汇总视图的模式,组织单个或个多个文件中的任务,并可以按要求进行组织及排序,显示出当前各个项目的情况详情;Agenda view有如下几种视图
\item 按日历格式显示任务,可以指定日期、日期范围、按日、周、月、年来进行显示任务清单
此模式目的是以当前日或者周来查看需要完成的工作清单;
\item 显示所有未完成的列表
显示所有文件或项目中标记为TODO的任务列表,没有时间
\item 匹配按表头、标签、TODO状态等来显示任务清单
\item 按timeline模式显示,按时间排序
\item 搜索模式,按检索词的检索结果来组织显示数据
\item 问题项目或问题任务模式,显示各个项目中出问题或暂停的项目
\item 按自定义视图显示内容

\item 在Agenda View模式中,查看或修改文件:
\end{enumerate}
当需要显示详细工作清单时,可以选择要查看的任务条目,点击 \textbf{TAB} 键,打开当前任务所在文件;如果需要修改,则在原文件中进行修改,修改完成后保存;切换到Agenda View中,按 \textbf{r} 键,则可以刷新修改后的内容;
在Agenda View中需要使用快捷键操作,具体可以参考: \url{http://orgmode.org/org.html#Agenda-commands}, 快捷键可以支持直接通过此视图修改原org文件中的状态;

\begin{enumerate}
\item 通过关键词搜索的方式显示
可以采用搜索关键词的方式显示,搜索关键词可以为: ‘+computer +wifi -ethernet‘,即包含computer, wifi,不包含ethernet的任务清单;
\item 显示Stuck项目
\end{enumerate}
\item 对Agenda View进行修改
\label{sec-2-2-4-2}
\begin{enumerate}
\item 新增新的命令
定义新的命令,来存储一些常用的搜索条件,定义需要显示的数据;此种方法可以按下面的代码形式,对Agenda Dispather进行定制:
\end{enumerate}
\lstset{basicstyle=\color{foreground}\small\mono,keywordstyle=\color{function}\bfseries\small\mono,identifierstyle=\color{doc}\small\mono,commentstyle=\color{comment}\small\itshape,stringstyle=\color{string}\small,showstringspaces=false,numbers=left,numberstyle=\color{preprocess},stepnumber=1,backgroundcolor=\color{background},tabsize=4,captionpos=t,breaklines=true,breakatwhitespace=true,showspaces=false,columns=flexible,frame=single,frameround=tttt,framesep=0pt,framerule=8pt,rulecolor=\color{background},fillcolor=\color{white},rulesepcolor=\color{comdil},framexleftmargin=10mm,language=SH,label= ,caption= }
\begin{lstlisting}
(setq org-agenda-custom-commands
           '(("x" agenda)
             ("y" agenda*)
             ("w" todo "WAITING")
             ("W" todo-tree "WAITING")
             ("u" tags "+boss-urgent")
             ("v" tags-todo "+boss-urgent")
             ("U" tags-tree "+boss-urgent")
             ("f" occur-tree "\\<FIXME\\>")
             ("h" . "HOME+Name tags searches") ; description for "h" prefix
             ("hl" tags "+home+Lisa")
             ("hp" tags "+home+Peter")
             ("hk" tags "+home+Kim")))
\end{lstlisting}
如上面定义了一些新的命令,即可以通过 \textbf{C-c a x} 打开agenda,通过 \textbf{C-c a w} 打开只包含 "WAITING"的状态的任务清单;
\begin{enumerate}
\item 对现有命令的显示模式进行修改,采用block view的方式显示,即在一个buffer中,显示多个block,一次性查看不同的内容要求;
\end{enumerate}
\lstset{basicstyle=\color{foreground}\small\mono,keywordstyle=\color{function}\bfseries\small\mono,identifierstyle=\color{doc}\small\mono,commentstyle=\color{comment}\small\itshape,stringstyle=\color{string}\small,showstringspaces=false,numbers=left,numberstyle=\color{preprocess},stepnumber=1,backgroundcolor=\color{background},tabsize=4,captionpos=t,breaklines=true,breakatwhitespace=true,showspaces=false,columns=flexible,frame=single,frameround=tttt,framesep=0pt,framerule=8pt,rulecolor=\color{background},fillcolor=\color{white},rulesepcolor=\color{comdil},framexleftmargin=10mm,language=SH,label= ,caption= }
\begin{lstlisting}
(setq org-agenda-custom-commands
           '(("h" "Agenda and Home-related tasks"
              ((agenda "")
               (tags-todo "home")
               (tags "garden")))
             ("o" "Agenda and Office-related tasks"
              ((agenda "")
               (tags-todo "work")
               (tags "office")))))
\end{lstlisting}
如上面定义,则在一个view buffer中,定义了不同的block,显示不同的内容, 如 \textbf{C-c a h} 则会显示三块内容,第一块显示agenda, 第二块显示包含"home"的todo标签的任务,第三个则是包含"garden"标签的任务;
\begin{enumerate}
\item org-mode包含一些可定义的命令,可以用于支持对自定义的命令进行特殊的定制,这些定义默认是通过全局有效使用,如果需要对个别命令,采用不同的配置要求,则可以针对不同的命令进行设备;
\end{enumerate}
\lstset{basicstyle=\color{foreground}\small\mono,keywordstyle=\color{function}\bfseries\small\mono,identifierstyle=\color{doc}\small\mono,commentstyle=\color{comment}\small\itshape,stringstyle=\color{string}\small,showstringspaces=false,numbers=left,numberstyle=\color{preprocess},stepnumber=1,backgroundcolor=\color{background},tabsize=4,captionpos=t,breaklines=true,breakatwhitespace=true,showspaces=false,columns=flexible,frame=single,frameround=tttt,framesep=0pt,framerule=8pt,rulecolor=\color{background},fillcolor=\color{white},rulesepcolor=\color{comdil},framexleftmargin=10mm,language=SH,label= ,caption= }
\begin{lstlisting}
(setq org-agenda-custom-commands
           '(("w" todo "WAITING"
              ((org-agenda-sorting-strategy '(priority-down))
               (org-agenda-prefix-format "  Mixed: ")))
             ("U" tags-tree "+boss-urgent"
              ((org-show-context-detail 'minimal)))
             ("N" search ""
              ((org-agenda-files '("~org/notes.org"))
               (org-agenda-text-search-extra-files nil)))))
\end{lstlisting}
如上面的定义,当执行 \textbf{C-c a w} 时显示只包含 "WAITING" 标签的任务,而再通过 \textbf{(org-agenda-sorting-strategy '(priority-down)} 来配置此view的排序条件为按优先级进行倒序排序;并增加 \textbf{Mixed:} 的前置;
另外,配置个性修改参数,可以为单命令级别进行配置,也可以为个别block进行单独配置命令,如:
\lstset{basicstyle=\color{foreground}\small\mono,keywordstyle=\color{function}\bfseries\small\mono,identifierstyle=\color{doc}\small\mono,commentstyle=\color{comment}\small\itshape,stringstyle=\color{string}\small,showstringspaces=false,numbers=left,numberstyle=\color{preprocess},stepnumber=1,backgroundcolor=\color{background},tabsize=4,captionpos=t,breaklines=true,breakatwhitespace=true,showspaces=false,columns=flexible,frame=single,frameround=tttt,framesep=0pt,framerule=8pt,rulecolor=\color{background},fillcolor=\color{white},rulesepcolor=\color{comdil},framexleftmargin=10mm,language=SH,label= ,caption= }
\begin{lstlisting}
(setq org-agenda-custom-commands
          '(("h" "Agenda and Home-related tasks"
             ((agenda)
              (tags-todo "home")
              (tags "garden"
                    ((org-agenda-sorting-strategy '(priority-up)))))
             ((org-agenda-sorting-strategy '(priority-down))))
            ("o" "Agenda and Office-related tasks"
             ((agenda)
              (tags-todo "work")
              (tags "office")))))
\end{lstlisting}
如上面命令,则是对 \textbf{C-c a h} 命令进行配置了整体 \textbf{((org-agenda-sorting-strategy '(priority-down)))}, 然而又单独对显示中的 \textbf{home} block进行配置 \textbf{((org-agenda-sorting-strategy '(priority-up)))} 的排序策略;
需要注意的是,参数中的值,可以是lisp语句,如果只是一个字符串,需要添加双引号;

\begin{enumerate}
\item 如果想要只针对某一种文本内容进行配置,则可以采用 \textbf{org-agenda-custom-commands-contexts} 进行配置,如:
\end{enumerate}
\lstset{basicstyle=\color{foreground}\small\mono,keywordstyle=\color{function}\bfseries\small\mono,identifierstyle=\color{doc}\small\mono,commentstyle=\color{comment}\small\itshape,stringstyle=\color{string}\small,showstringspaces=false,numbers=left,numberstyle=\color{preprocess},stepnumber=1,backgroundcolor=\color{background},tabsize=4,captionpos=t,breaklines=true,breakatwhitespace=true,showspaces=false,columns=flexible,frame=single,frameround=tttt,framesep=0pt,framerule=8pt,rulecolor=\color{background},fillcolor=\color{white},rulesepcolor=\color{comdil},framexleftmargin=10mm,language=SH,label= ,caption= }
\begin{lstlisting}
(setq org-agenda-custom-commands-contexts
           '(("o" (in-mode . "message-mode"))))
\end{lstlisting}
如上面命令,则只针对 \textbf{message-mode} 有效;
还可以将某一命令中,引用其它命令进行操作,如:
\lstset{basicstyle=\color{foreground}\small\mono,keywordstyle=\color{function}\bfseries\small\mono,identifierstyle=\color{doc}\small\mono,commentstyle=\color{comment}\small\itshape,stringstyle=\color{string}\small,showstringspaces=false,numbers=left,numberstyle=\color{preprocess},stepnumber=1,backgroundcolor=\color{background},tabsize=4,captionpos=t,breaklines=true,breakatwhitespace=true,showspaces=false,columns=flexible,frame=single,frameround=tttt,framesep=0pt,framerule=8pt,rulecolor=\color{background},fillcolor=\color{white},rulesepcolor=\color{comdil},framexleftmargin=10mm,language=SH,label= ,caption= }
\begin{lstlisting}
(setq org-agenda-custom-commands-contexts
           '(("o" "r" (in-mode . "message-mode"))))
\end{lstlisting}
\item Agenda View 导出
\label{sec-2-2-4-3}
Agenda view可以导出为text, html, pdf, postscript, icalendar格式;
\end{enumerate}
\subsection{Tools in Emacs}
\label{sec-2-3}
\begin{itemize}
\item Ielm Elisp编译器,M-x ielm
\item 使配置生效,M-x eval-buffer
\item 宏
\item 
\end{itemize}
\subsection{Emacs Wanderlust}
\label{sec-2-4}
\subsubsection{Wanderlust Install}
\label{sec-2-4-1}
在Archlinux上有包的管理:因此直接通过sudo命令即可以直接安装;
\lstset{basicstyle=\color{foreground}\small\mono,keywordstyle=\color{function}\bfseries\small\mono,identifierstyle=\color{doc}\small\mono,commentstyle=\color{comment}\small\itshape,stringstyle=\color{string}\small,showstringspaces=false,numbers=left,numberstyle=\color{preprocess},stepnumber=1,backgroundcolor=\color{background},tabsize=4,captionpos=t,breaklines=true,breakatwhitespace=true,showspaces=false,columns=flexible,frame=single,frameround=tttt,framesep=0pt,framerule=8pt,rulecolor=\color{background},fillcolor=\color{white},rulesepcolor=\color{comdil},framexleftmargin=10mm,language=sh,label= ,caption= }
\begin{lstlisting}
sudo pacman -S wanderlust
\end{lstlisting}
安装完成后,它可能会在/usr/share/emacs/site-lisp/wl下生成对应的el配置文件;
可以将配置文件拷到对应\textasciitilde{}/.emacs.d/site-list/wl下;
接着需要安装它的相应的依赖包:elmo, bbdb, semi, w3m, 如果使用ssl协议还需要下载ssl.el文件;
这些包可以在一些相应的网站可以下载:如\url{http://www.emacswiki.org/上下载};

\subsubsection{Wanderlust Configuration}
\label{sec-2-4-2}
下载我的配置文件到你的文件夹中:\url{https://github.com/lujianmei/.emacs.d};
找到\textasciitilde{}/.emacs.d/user/kevin/init-wl.el\textasciitilde{}文件下载;
\begin{itemize}
\item 配置用户名,目录等相关信息
\begin{itemize}
\item 按文件中的目录信息配置相关的存储目录,默认为\textasciitilde{}/mails,如果不需要修改则可以不用修改;
\end{itemize}
\item 配置imap目录,与邮箱web端的目录结构相同
\begin{itemize}
\item 参照\textasciitilde{}/.emacs.d/user/kevin/folders文件,配置与生产环境相同的目录结构;
\item 可以直接通过邮箱服务端配置好邮件的过滤功能,然后直接在wl中进行下载查看;
\end{itemize}
\item 配置登录密码,邮件发送密码
\begin{itemize}
\item init-wl.el文件中已经默认配置好了passwd的文件目录,用于存储imap, stmp的加密后的密码信息的;
\item 因此可以将folders文件按要求放到对应的目录下,然后在登录邮箱,并成功发送邮件后,执行:M-x elmo-passwd-alist-save方法,它即会自动将密码信息写入此文件中;
\item 下次即不再要求通过密码校验了;
\end{itemize}
\item 配置签名文件
\begin{itemize}
\item 可以在init-wl.el文件中找到对应的signature文件的目录,因此在对应的地方新建一个文件,然后将签名内容拷进去;
\item 则可以在发送邮件时自动生成对应的签名在后面;
\end{itemize}
\item 配置邮件附件打开方式
\begin{itemize}
\item 参照\textasciitilde{}/.emacs.d/user/kevin/mailcap文件,配置当前操作系统下的用来查看附件文件的方式;
\item 如下例子:
\lstset{basicstyle=\color{foreground}\small\mono,keywordstyle=\color{function}\bfseries\small\mono,identifierstyle=\color{doc}\small\mono,commentstyle=\color{comment}\small\itshape,stringstyle=\color{string}\small,showstringspaces=false,numbers=left,numberstyle=\color{preprocess},stepnumber=1,backgroundcolor=\color{background},tabsize=4,captionpos=t,breaklines=true,breakatwhitespace=true,showspaces=false,columns=flexible,frame=single,frameround=tttt,framesep=0pt,framerule=8pt,rulecolor=\color{background},fillcolor=\color{white},rulesepcolor=\color{comdil},framexleftmargin=10mm,language=sh,label= ,caption= }
\begin{lstlisting}
application/pdf; okular  %s
application/msword; catdoc %s
application/octet-stream; et  %s
application/octet-stream; wpp  %s
application/octet-stream; wps  %s
application/*; xdg-open  %s
image/*; ristretto %s
text/html; chromium %s
text/*; emacsclient -c %s
video/*; xdg-open %s
audio/*; xdg-open %s
application/x-rar; xarchiver %s
application/x-zip; xarchiver %s
application/x-tar; xarchiver %s
\end{lstlisting}
\end{itemize}
\end{itemize}
\subsection{Eshell}
\label{sec-2-5}

\textbf{*}

\subsection{Tramp}
\label{sec-2-6}
\subsection{Auctex}
\label{sec-2-7}

**
\subsection{Latex}
\label{sec-2-8}
\subsection{Plantuml}
\label{sec-2-9}
\url{http://www.plantuml.com/}
\begin{itemize}
\item 安装依赖
在archlinux的aur中找到plantuml进行安装;
\item 使用,在需要生成的图的位置代码用如何代码框起来
\end{itemize}
\lstset{basicstyle=\color{foreground}\small\mono,keywordstyle=\color{function}\bfseries\small\mono,identifierstyle=\color{doc}\small\mono,commentstyle=\color{comment}\small\itshape,stringstyle=\color{string}\small,showstringspaces=false,numbers=left,numberstyle=\color{preprocess},stepnumber=1,backgroundcolor=\color{background},tabsize=4,captionpos=t,breaklines=true,breakatwhitespace=true,showspaces=false,columns=flexible,frame=single,frameround=tttt,framesep=0pt,framerule=8pt,rulecolor=\color{background},fillcolor=\color{white},rulesepcolor=\color{comdil},framexleftmargin=10mm,language=LISP,label= ,caption= }
\begin{lstlisting}
 #+begin_src plantuml :file some_filename.png :cmdline -r -s 0.8
<context of ditaa source goes here>
 #+end_src
\end{lstlisting}
\subsection{Graphviz}
\label{sec-2-10}
\url{http://www.graphviz.org/}
\begin{itemize}
\item 安装依赖
\end{itemize}
\lstset{basicstyle=\color{foreground}\small\mono,keywordstyle=\color{function}\bfseries\small\mono,identifierstyle=\color{doc}\small\mono,commentstyle=\color{comment}\small\itshape,stringstyle=\color{string}\small,showstringspaces=false,numbers=left,numberstyle=\color{preprocess},stepnumber=1,backgroundcolor=\color{background},tabsize=4,captionpos=t,breaklines=true,breakatwhitespace=true,showspaces=false,columns=flexible,frame=single,frameround=tttt,framesep=0pt,framerule=8pt,rulecolor=\color{background},fillcolor=\color{white},rulesepcolor=\color{comdil},framexleftmargin=10mm,language=SH,label= ,caption= }
\begin{lstlisting}
sudo pacman -S graphviz
\end{lstlisting}
\begin{itemize}
\item 使用,在需要生成的图的位置代码用如何代码框起来
\end{itemize}
\lstset{basicstyle=\color{foreground}\small\mono,keywordstyle=\color{function}\bfseries\small\mono,identifierstyle=\color{doc}\small\mono,commentstyle=\color{comment}\small\itshape,stringstyle=\color{string}\small,showstringspaces=false,numbers=left,numberstyle=\color{preprocess},stepnumber=1,backgroundcolor=\color{background},tabsize=4,captionpos=t,breaklines=true,breakatwhitespace=true,showspaces=false,columns=flexible,frame=single,frameround=tttt,framesep=0pt,framerule=8pt,rulecolor=\color{background},fillcolor=\color{white},rulesepcolor=\color{comdil},framexleftmargin=10mm,language=LISP,label= ,caption= }
\begin{lstlisting}
#+begin_src dot :file some_filename.png :cmdline -Kdot -Tpng
   <context of graphviz source goes here>
#+end_src
\end{lstlisting}
\subsection{Diaat}
\label{sec-2-11}
\begin{itemize}
\item 安装依赖
\end{itemize}
\lstset{basicstyle=\color{foreground}\small\mono,keywordstyle=\color{function}\bfseries\small\mono,identifierstyle=\color{doc}\small\mono,commentstyle=\color{comment}\small\itshape,stringstyle=\color{string}\small,showstringspaces=false,numbers=left,numberstyle=\color{preprocess},stepnumber=1,backgroundcolor=\color{background},tabsize=4,captionpos=t,breaklines=true,breakatwhitespace=true,showspaces=false,columns=flexible,frame=single,frameround=tttt,framesep=0pt,framerule=8pt,rulecolor=\color{background},fillcolor=\color{white},rulesepcolor=\color{comdil},framexleftmargin=10mm,language=SH,label= ,caption= }
\begin{lstlisting}
sudo pacman -S ditaa
\end{lstlisting}
\begin{itemize}
\item 使用,在需要生成的图的位置代码用如何代码框起来
\end{itemize}
\lstset{basicstyle=\color{foreground}\small\mono,keywordstyle=\color{function}\bfseries\small\mono,identifierstyle=\color{doc}\small\mono,commentstyle=\color{comment}\small\itshape,stringstyle=\color{string}\small,showstringspaces=false,numbers=left,numberstyle=\color{preprocess},stepnumber=1,backgroundcolor=\color{background},tabsize=4,captionpos=t,breaklines=true,breakatwhitespace=true,showspaces=false,columns=flexible,frame=single,frameround=tttt,framesep=0pt,framerule=8pt,rulecolor=\color{background},fillcolor=\color{white},rulesepcolor=\color{comdil},framexleftmargin=10mm,language=LISP,label= ,caption= }
\begin{lstlisting}
 #+begin_src ditaa :file some_filename.png :cmdline -r -s 0.8
<context of ditaa source goes here>
 #+end_src
\end{lstlisting}
\subsection{TernJs}
\label{sec-2-12}
\subsection{Sunrise Commander}
\label{sec-2-13}
\begin{itemize}
\item 说明
Sunrise是一款类似于dired的命令行的文件管理器,具有大量命令行操作工具;
\item 使用方法:显示当前及其子文件夹下所有文件
press C-c C-f and type:  -not -type d
\end{itemize}

\subsection{GNUS with Offlineimap and mu4e and msmtp}
\label{sec-2-14}
\subsubsection{Offlineimap}
\label{sec-2-14-1}
\begin{enumerate}
\item 安装
\label{sec-2-14-1-1}
\lstset{basicstyle=\color{foreground}\small\mono,keywordstyle=\color{function}\bfseries\small\mono,identifierstyle=\color{doc}\small\mono,commentstyle=\color{comment}\small\itshape,stringstyle=\color{string}\small,showstringspaces=false,numbers=left,numberstyle=\color{preprocess},stepnumber=1,backgroundcolor=\color{background},tabsize=4,captionpos=t,breaklines=true,breakatwhitespace=true,showspaces=false,columns=flexible,frame=single,frameround=tttt,framesep=0pt,framerule=8pt,rulecolor=\color{background},fillcolor=\color{white},rulesepcolor=\color{comdil},framexleftmargin=10mm,language=sh,label= ,caption= }
\begin{lstlisting}
sudo pacman -Ss offlineimap
\end{lstlisting}
\item 配置
\label{sec-2-14-1-2}
Linux 配置
\lstset{basicstyle=\color{foreground}\small\mono,keywordstyle=\color{function}\bfseries\small\mono,identifierstyle=\color{doc}\small\mono,commentstyle=\color{comment}\small\itshape,stringstyle=\color{string}\small,showstringspaces=false,numbers=left,numberstyle=\color{preprocess},stepnumber=1,backgroundcolor=\color{background},tabsize=4,captionpos=t,breaklines=true,breakatwhitespace=true,showspaces=false,columns=flexible,frame=single,frameround=tttt,framesep=0pt,framerule=8pt,rulecolor=\color{background},fillcolor=\color{white},rulesepcolor=\color{comdil},framexleftmargin=10mm,language=sh,label= ,caption= }
\begin{lstlisting}
[general]
ui = TTYUI
accounts = TRS
pythonfile = ~/.mutt/source/offlineimap.py
fsync = False

[Account TRS]
localrepository = TRS-Local
remoterepository = TRS-Remote
status_backend = sqlite
postsynchook = notmuch new
# Minutes between syncs
autorefresh = 5
# Number of quick-syncs between autorefreshes. Quick-syncs do not update if the
# only changes were to IMAP flags
quick = 10

[Repository TRS-Local]
type = Maildir
localfolders = ~/.mutt/mails/lu.jianmei/
nametrans = lambda foldername: foldername.decode('imap4-utf-7').encode('utf-8')
#nametrans = lambda folder : {'drafts':   '草稿箱',
#                            'inbox':     'Inbox',
#                            'sent':     '已发送',
##                            'flagged':  '[TRS]/Starred',
#                            'trs':  'trs',
#                            'haier':  'haier',
#                            'pm':  'pm',
##                            'trash':    '[TRS]/Bin',
#                            'archive':  'All Mail',
##                           }.get(folder.decode('imap4-utf-7').encode('utf-8'), folder.decode('imap4-utf-7').encode('utf-8'))
#                           }.get(folder.decode('imap4-utf-7').encode('utf-8'), folder.decode('imap4-utf-7').encode('utf-8'))

[Repository TRS-Remote]
sslcacertfile=/etc/ssl/certs/ca-certificates.crt
maxconnections = 2
type = IMAP
auth = on
ssl = yes
#reference = Mail
remotehost = imap.qiye.163.com
remoteuser = lu.jianmei@trs.com.cn
remoteport = 993

remotepasseval = get_gpg_pass(keyfile="/home/kevin/.mutt/.my-pwds.gpg")
realdelete = no
startdate = 2015-04-01
# solve foldername encoding, for supporting chinese foldername in remote server
# foldername: foldername.decode('imap4-utf-7').encode('utf-8')
#nametrans = lambda folder: foldername.decode('imap4-utf-7').encode('utf-8') : {'Drafts':     'drafts',
nametrans = lambda folder: folder.decode('imap4-utf-7').encode('utf-8')

#folderfilter = lambda folder: folder not in ['/Bin', '/Spam','[TRS]/akamai','[TRS]/errors','[TRS]/me','[TRS]/nagios']
folderfilter = lambda folder: folder in ['INBOX','已发送','草稿箱', 'trs','haier','pm']
# Instead of closing the connection once a sync is complete, offlineimap will
# send empty data to the server to hold the connection open. A value of 60
# attempts to hold the connection for a minute between syncs (both quick and
# autorefresh).This setting has no effect if autorefresh and holdconnectionopen
# are not both set.
keepalive = 60
# OfflineIMAP normally closes IMAP server connections between refreshes if
# the global option autorefresh is specified.  If you wish it to keep the
# connection open, set this to true. This setting has no effect if autorefresh
# is not set.
holdconnectionopen = yes
\end{lstlisting}

Mac配置
\lstset{basicstyle=\color{foreground}\small\mono,keywordstyle=\color{function}\bfseries\small\mono,identifierstyle=\color{doc}\small\mono,commentstyle=\color{comment}\small\itshape,stringstyle=\color{string}\small,showstringspaces=false,numbers=left,numberstyle=\color{preprocess},stepnumber=1,backgroundcolor=\color{background},tabsize=4,captionpos=t,breaklines=true,breakatwhitespace=true,showspaces=false,columns=flexible,frame=single,frameround=tttt,framesep=0pt,framerule=8pt,rulecolor=\color{background},fillcolor=\color{white},rulesepcolor=\color{comdil},framexleftmargin=10mm,language=sh,label= ,caption= }
\begin{lstlisting}
[general]
ui = TTYUI
accounts = TRS
pythonfile = ~/.mutt/source/offlineimap.py
fsync = False

[Account TRS]
localrepository = TRS-Local
remoterepository = TRS-Remote
status_backend = sqlite
#postsynchook = notmuch new
# Minutes between syncs, use mu4e to refresh
autorefresh = 0
# Number of quick-syncs between autorefreshes. Quick-syncs do not update if the
# only changes were to IMAP flags
quick = 10

[Repository TRS-Local]
type = Maildir
#localfolders = ~/.mutt/mails/lu.jianmei/
localfolders = ~/Maildir/lu.jianmei/
#nametrans = lambda foldername: foldername.decode('imap4-utf-7').encode('utf-8')
#nametrans = lambda folder : {'drafts':   '草稿箱',
#                            'inbox':     'Inbox',
#                            'sent':     '已发送',
##                            'flagged':  '[TRS]/Starred',
#                            'trs':  'trs',
#                            'haier':  'haier',
#                            'pm':  'pm',
##                            'trash':    '[TRS]/Bin',
#                            'archive':  'All Mail',
##                           }.get(folder.decode('imap4-utf-7').encode('utf-8'), folder.decode('imap4-utf-7').encode('utf-8'))
#                           }.get(folder.decode('imap4-utf-7').encode('utf-8'), folder.decode('imap4-utf-7').encode('utf-8'))

[Repository TRS-Remote]
#Sslcacertfile=/etc/ssl/certs/ca-certificates.crt
sslcacertfile =  /Users/kevin/.emacs.d/ca-bundle.crt
maxconnections = 2
type = IMAP
auth = on
ssl = on
#reference = Mail
remotehost = imap.qiye.163.com
remoteuser = lu.jianmei@trs.com.cn

# imap protocol port: 993 for ssl, 143 for none ssl
remoteport = 993

#remotepasseval = get_gpg_pass(keyfile="/home/kevin/.mutt/.my-pwds.gpg")
realdelete = no


#folderfilter = lambda folder: folder not in ['/Bin', '/Spam','[TRS]/akamai','[TRS]/errors','[TRS]/me','[TRS]/nagios']
# use offlineimap --info can know the remote folders. (following two unknow name is 已发送 and 草稿箱 and 已删除)
folderfilter = lambda folder: folder in ['INBOX','trs','haier','pm','me','&XfJT0ZAB-','&g0l6P3ux-','Sent','&XfJSIJZk-']

startdate = 2015-01-01
# solve foldername encoding, for supporting chinese foldername in remote server
# foldername: foldername.decode('imap4-utf-7').encode('utf-8')
#nametrans = lambda folder: foldername.decode('imap4-utf-7').encode('utf-8') : {'Drafts':     'drafts',
#nametrans = lambda x: 'INBOX.' + x
nametrans = lambda folder: folder.decode('imap4-utf-7').encode('utf-8')
###nametrans = lambda folder : {'草稿箱':     'drafts',
###                            'INBOX':  'inbox',
###                            '已发送':  'sent',
####                            '[TRS]/Starred':    'flagged',
###                            'trs':    'trs',
###                            'haier':    'haier',
###                            'pm':    'pm',
####                            'Bin':        'trash',
###                            'All Mail':   'archive',
####                           }.get(folder.decode('imap4-utf-7').encode('utf-8'), folder.decode('imap4-utf-7').encode('utf-8'))
###                           }.get(folder, folder.decode('imap4-utf-7').encode('utf-8'))

# Instead of closing the connection once a sync is complete, offlineimap will
# send empty data to the server to hold the connection open. A value of 60
# attempts to hold the connection for a minute between syncs (both quick and
# autorefresh).This setting has no effect if autorefresh and holdconnectionopen
# are not both set.
keepalive = 60
# OfflineIMAP normally closes IMAP server connections between refreshes if
# the global option autorefresh is specified.  If you wish it to keep the
# connection open, set this to true. This setting has no effect if autorefresh
# is not set.
holdconnectionopen = yes
\end{lstlisting}
\item 配置安全密码
\label{sec-2-14-1-3}
\end{enumerate}
\subsubsection{Mu4e}
\label{sec-2-14-2}
mu4e基于mu开发,直接通过maildir格式文件进行读取邮件内容;因此可以通过offlineimap进行邮件下载,然后通过mu4e进行邮件读取及发送邮件即可;
官方网站:[\url{http://www.djcbsoftware.nl/code/mu}]
\begin{enumerate}
\item 安装
\label{sec-2-14-2-1}
\lstset{basicstyle=\color{foreground}\small\mono,keywordstyle=\color{function}\bfseries\small\mono,identifierstyle=\color{doc}\small\mono,commentstyle=\color{comment}\small\itshape,stringstyle=\color{string}\small,showstringspaces=false,numbers=left,numberstyle=\color{preprocess},stepnumber=1,backgroundcolor=\color{background},tabsize=4,captionpos=t,breaklines=true,breakatwhitespace=true,showspaces=false,columns=flexible,frame=single,frameround=tttt,framesep=0pt,framerule=8pt,rulecolor=\color{background},fillcolor=\color{white},rulesepcolor=\color{comdil},framexleftmargin=10mm,language=sh,label= ,caption= }
\begin{lstlisting}
yaourt -S mu
\end{lstlisting}
mac上的安装
\lstset{basicstyle=\color{foreground}\small\mono,keywordstyle=\color{function}\bfseries\small\mono,identifierstyle=\color{doc}\small\mono,commentstyle=\color{comment}\small\itshape,stringstyle=\color{string}\small,showstringspaces=false,numbers=left,numberstyle=\color{preprocess},stepnumber=1,backgroundcolor=\color{background},tabsize=4,captionpos=t,breaklines=true,breakatwhitespace=true,showspaces=false,columns=flexible,frame=single,frameround=tttt,framesep=0pt,framerule=8pt,rulecolor=\color{background},fillcolor=\color{white},rulesepcolor=\color{comdil},framexleftmargin=10mm,language=sh,label= ,caption= }
\begin{lstlisting}
# install email client related
brew install gnutls
#install email related
brew install mu --with-emacs
brew install offlineimap msmtp
brew install curl --with-openssl && brew link curl —forc
brew install html2text w3m
\end{lstlisting}
\item 配置
\label{sec-2-14-2-2}
配置直接通过emacs中的init-mu4e.el文件进行配置;
完成mu4e的配置后,需要创建\textasciitilde{}/.authinfo文件,存储所需要的服务器用户名密码等信息;
\lstset{basicstyle=\color{foreground}\small\mono,keywordstyle=\color{function}\bfseries\small\mono,identifierstyle=\color{doc}\small\mono,commentstyle=\color{comment}\small\itshape,stringstyle=\color{string}\small,showstringspaces=false,numbers=left,numberstyle=\color{preprocess},stepnumber=1,backgroundcolor=\color{background},tabsize=4,captionpos=t,breaklines=true,breakatwhitespace=true,showspaces=false,columns=flexible,frame=single,frameround=tttt,framesep=0pt,framerule=8pt,rulecolor=\color{background},fillcolor=\color{white},rulesepcolor=\color{comdil},framexleftmargin=10mm,language=sh,label= ,caption= }
\begin{lstlisting}
machine smtp.qiye.163.com login username password yourpassword
\end{lstlisting}
\end{enumerate}

\subsubsection{msmtp}
\label{sec-2-14-3}
msmtp工具用于发送邮件,mutt的基本配置内容可以参照\url{https://github.com/lujianmei/.mutt} 上的配置进行处理;
\lstset{basicstyle=\color{foreground}\small\mono,keywordstyle=\color{function}\bfseries\small\mono,identifierstyle=\color{doc}\small\mono,commentstyle=\color{comment}\small\itshape,stringstyle=\color{string}\small,showstringspaces=false,numbers=left,numberstyle=\color{preprocess},stepnumber=1,backgroundcolor=\color{background},tabsize=4,captionpos=t,breaklines=true,breakatwhitespace=true,showspaces=false,columns=flexible,frame=single,frameround=tttt,framesep=0pt,framerule=8pt,rulecolor=\color{background},fillcolor=\color{white},rulesepcolor=\color{comdil},framexleftmargin=10mm,language=sh,label= ,caption= }
\begin{lstlisting}
brew install msmtp
\end{lstlisting}


\section{Shot-key}
\label{sec-3}
\subsection{Base}
\label{sec-3-1}
\begin{center}
\begin{tabular}{lll}
\hline
Move & C-S-down & 往下移动行\\
 & C-e & 去到行尾\\
 & M-up/down & html模式当中, 按标签对上下移动\\
 & C-a & 返回到行首\\
 & C-l & 调整当前光标所在行为屏幕最上面或中间或最下面\\
 & M-> & 跳转到页面最后\\
 & M-< & 跳转到文件最头\\
 & M-n/p & 跳转块,跳转到下一个空行;\\
 & C-c C-f & Go to next line and make the point at the end of this line\\
 & C-c C-b & Back to above line make the point at the end o fthis liner\\
 & C-S n/p/b/f & 一次性移动5格\\
 & M-i & 返回到本行的缩进位置\\
 & C-v & 向下移动一页\\
 & M-v & 向上移动一页\\
\hline
Select & C-> & 向下选择多个光标\\
 & C-< & 向上选择多个光标\\
 & C-c h & 全选\\
 & C-RET & 进入矩形编辑,然后C-n/p可以选择\\
 & C-S 鼠标点击 & 通过鼠标点击选择多个光标\\
\hline
Windows & C-x 1 & 只显示当前窗口,关闭其它窗口\\
 & C-x 2 & 上下方式打开一个新窗口\\
 & C-x 3 & 左右方式打开一个新窗口\\
 & C-x o & 选择窗口\\
 & C-x 0 & 关闭当前窗口…\\
 & C-x 5 & 切换当前buffer到指定的windows中\\
 &  & \\
\hline
Edit & C-c d & 复制当前行\\
 & C-c b & 新建一个文件并打开buffer\\
 & C-c c & 注释/取消注释\\
 & M-; & 选择,然后打注释\\
 & M-RET & 下面新建一行并自动缩进\\
 & C-o & 新建一行并自动缩进,但光标不变化\\
 & C-k & 删除光标后面的内容,html模式中可直接删除整个tag集\\
 & C-S-k & 不管光标在哪,删除此行且光标移动到缩进首\\
 & M-w & 复制当前行,不用选择也不用移动到行首\\
 & C-h & 删除已经选择的内容,删除内容\\
 & C-S-i & 缩进已经选择的或当前行\\
 & M-j & 将上一行缩进到本行后面\\
 & C-; & 当前系统剪贴版\\
 & M-u & 大写转换\\
 & M-q & 对长的行进行自动折行处理\\
 & C-h & 删除退格键\\
 & C-y & 粘贴内容\\
 & C-x C-y & 选择性粘贴内容,打开剪贴板\\
 & C-x C-q & 只读与非只读之间切换\\
 &  & \\
\hline
Search/Replace & C-s & 往后搜索\\
 & C-s M-i & 打开小窗口进行搜索\\
 & C-r & 往前搜索\\
 & M-\% & 查找替换, y替换,n不替换,q退出,!替换后面所有\\
\hline
Register & C-x r SPC [number/charactor] & 将当前光标所在位置注册到Register中\\
 & C-x r j [number/charactor] & 跳转到register对应记录所在的位置\\
 & C-x r s [number/charactor] & 将选择的区域存储到register中\\
 & C-x r i [number/charactor] & 将register中的对应内容插入到当前光标位置处\\
 & C-u C-x r s [number/charactor] & 将选择的区域剪切到register中\\
\hline
Narrow & C-x n n & 将选定区域获取新编辑窗口\\
 & C-x n p & 将当前页面获取进入narrow窗口\\
 & C-x n d & 将当前方法获取进入narrow窗口\\
 & C-x n w & 取消narrow\\
\hline
View mode & M-x view-mode & 进入查看模式\\
 & SPC & 在查看模式向下滚动\\
 & S-SPC & 在查看模式向上滚动\\
 & q & 退出查看模式,并回到启动viewmode的位置\\
 & e & 退出查看模式,并保持当前的光标位置\\
\hline
Follow mode & C-x 3 M-x follow-mode & 打开一个新窗口,并启动follow-mode\\
 & M-x follow-mode & 关闭follow-mode\\
\hline
Mark & C-M-, & 将当前行设置一个mark,可以通过C-M-<进行退回\\
 & C-M-< & 退回到上一个mark的行,用于快速返回\\
 & C-M-> & 取消所有的mark,用于对mark进行初使化\\
\hline
Emacs & C-x r q & 快速退出emacs\\
 & C-x C-c & 退出emacs标准版\\
 & C-c C-s & 保存当前文件\\
 & M-[ & 扩大当前窗口\\
 & M-] & 缩小当前窗口\\
 & C-x C-+ & 放大当前buffer字体\\
 & C-x C-0 & 返回原来buffer字体大小(zoom-frm-in/out可以对整个frm的字体进行放大缩小,zoom-in/out功能相同)\\
 & C-x C-- & 切换当前窗口内buffer的顺序\\
 & C-x - & 切换当前窗口之间的结构,横向切换为纵向,反之\\
 & C-x C-w & 另存为\\
 & C-x RET & 放大窗口/缩小窗口\\
 & C-x b & 切换文件\\
\hline
Smart selection & C-' & 智能选择区域,适用于如csc, js, html等代码模式,org模式则为打开另一个org模式文件\\
 &  & \\
\hline
HTML-edit & M-up/down & Tags成对移动\\
\hline
Shell & C-z & 打开shell-mode\\
 &  & \\
\hline
Eval & M-: & 打开eval功能,查找emacs接口文档\\
 &  & \\
\hline
Help & M-x helm-M-x & 查看command键映射\\
 &  & \\
 &  & \\
 &  & \\
\hline
\end{tabular}
\end{center}

\subsection{Org-mode}
\label{sec-3-2}
\begin{center}
\begin{tabular}{lll}
分类 & 快捷键 & 说明\\
\hline
org-mode & C-RET & 加入同级别索引\\
 & M-RET & 加入同级别的列表\\
 & C-c C-t & 设置TODO标签\\
 & M-left/M-right & 修改任务等级,子任务不跟着变化\\
 & M-S-up/down & 调整此任务节点等级,子任务跟着变化\\
 & C-c - & 更换列表标记(循环)\\
 & M-S-RET & 新增一个子项\\
 & M-up/M-down & 调整此任务节点的顺序\\
\hline
outline & C-c C-p & 上一个标题\\
 & C-c C-n & 下一下\\
 & C-c C-f & 同一级的上一个\\
 & C-c C-b & 同一级的下一个\\
 & C-c C-u & 回到上一级标题\\
 & C-c C-j & 跳转标题\\
\hline
column & C-c C-x C-c & 打开column视图模式\\
 & r & 刷新\\
 & q & 退出\\
 & <left> <right> <up> <down> & 视图之间跳转\\
 & v & 查看属性完整值\\
 & C-c C-x i & 插入column视图在文件中\\
 & C-c C-x C-u & 更新column视图中的值,需要进入表格中执行\\
 & C-u C-c C-x C-u & 更新此文件中所有的column视图内容\\
 &  & \\
 &  & \\
 &  & \\
\hline
Property & C-c C-x p & 设置一个属性\\
 & C-c C-x p COLUMN & 设置column,内容可以为\%25ITEM 10\%ITEM 5\%TODO 30\%SCEDULE 30\%DEADLINE\\
\hline
Tags & C-c C-c C-c & 打开tag选择窗口,然后通过字母索引选择tag\\
 & SPC & 清除所有tag\\
 & C-c C-c & 可以直接输入tag的单词直接进行选择\\
 & C-c C-x C-c & 打开列展示视图\\
 & q & 退出列视图\\
\hline
Planning & C-c . & 设置时间\\
 & S-left/S-right & 在日历中选择时间\\
 & M-n/M-p & 设置任务的优先级\\
 & C-c C-s & 设置任务开始时间, schedlued\\
 & C-c C-d & 设置任务结束时间,deadline\\
 & C-c / d & 显示警告天数之内的Deadline任务\\
 & C-u C-c / d & 显示所有的deadline任务\\
 & C-1 C-c / d & 查看一天之内接近的deadline任务列表\\
 & C-c / b & 查看指定日期之前的deadline, schedule任务列表\\
 & C-c / a & 查看指定日期之后的deadline, schedule任务列表\\
 & C-c . & 插入时间(Timestamps)\\
 & S-left/right & 光标在时间上时,用于往前一天往后一天调整\\
\hline
Clocking & C-c C-x C-i & 开始clock\\
 & C-c C-x C-o & 退出clock\\
 & C-c C-x C-r & 插入clock table\\
 & C-c C-x ; & Start a count down time\\
\hline
Agenda & C-c a & 打开agenda view, 然后根据显示视图进行选择性显示\\
 & C-c [ & 添加当前文件进入agenda-view-file中\\
 & C-c ] & 删除当前文件从agenda-view-file中\\
 & C-c C-x < & 强制限制只使用当前文件或当前节点来显示agenda-view\\
 & C-c C-x > & 取消以上限制\\
 & C-c a t & 显示TODO列表\\
 & C-c a T & 可以指定要显示的状态列表,多个状态使用"竖线"隔开显示\\
 & C-c a m & 匹配 tags and properties\\
 & C-c a M & 匹配搜索的tag\\
 & C-c a L & 采用timeline的方式显示此项目,只能在一个单文件上执行此操作\\
 & C-c a s & 按搜索关键查询\\
 & C-c a \# & 列出项目暂停的任务\\
 &  & \\
 &  & \\
 & C-c C-w & 导出文件\\
\hline
Agenda column & C-c C-x C-c & 打开column模式在agenda view中\\
 &  & \\
 &  & \\
 &  & \\
\hline
Capture & C-c c & 打开capture\\
 &  & \\
 &  & \\
\hline
Export & C-<f12> & 一次性生成所有目录的org文件为html文件,发布配置见.emacs.d中的配置目录\\
 & C-c C-e & 导出\\
\hline
Tables & C-c - & 在下面添加水平线\\
 & C-c RET & 添加水平线并跳转到下一行\\
 & C-m & 在本列下面添加一行\\
 & M-S-Right & 在本列后面添加一列\\
 & M-S-Down & 在本行上面添加一行\\
 & M-S-Left & 删除本列\\
 & M-S-UP & 删除本行\\
 & M-left/right & 移动列\\
 & M-Up/Down & 移动行\\
 & C-c C-c & 重新定义表格\\
 & C-c ` & 修改隐藏的表格中的内容\\
\end{tabular}
\end{center}

\subsection{宏}
\label{sec-3-3}
\begin{center}
\begin{tabular}{lll}
\hline
宏 & C-x ( & 开始录制宏\\
 & C-x ) & 结束录制宏\\
 & C-x e & 使用宏\\
 & C-u & 重复使用宏,C-u 100 C-x e重复100次\\
 & M-x name-last-kbd-macro & 保存宏,可以在其它地方通过M-x调用此保存好的宏\\
\hline
 &  & \\
\end{tabular}
\end{center}

\subsection{Dired}
\label{sec-3-4}
\begin{longtable}{l|l|l}
\caption{Dired快捷键}
\\
类别 & 快捷键 & 描述\\
\hline
\endhead
\hline\multicolumn{3}{r}{Continued on next page} \\
\endfoot
\endlastfoot
基本 & C-x d & 启动dired\\
 & ; & 切换View-mode与Dired-mode,View-mode可以通过首字母定位文件名,Dired-mode下可以使用快捷键\\
\hline
Dired-mode & n/p & 上一个,下一个\\
 &  & \\
 & \$ & 隐藏/显示目录结构\\
 & p & 上一个文件夹/文件\\
 & n & 下一个文件夹/文件\\
 & q & 返回目录\\
 & o & 另一个窗口打开文件\\
 & g & 刷新当前目录\\
 & l & 列出当前详细信息\\
\hline
 & m & 标记当前文件夹/文件\\
 & t & 标记所有\\
 & u & 取消标记\\
 & d & 标记为删除\\
 & R & 重命名\\
\hline
 & X & 删除\\
 & k & 移动到回收站\\
 & R & 移动或重命名\\
 & C & 复制\\
 & + & 新建文件夹\\
 & C-x C-f & 新建文件\\
\hline
 & M & 改变权限\\
 & O & 改变用户\\
\hline
 & M-g & 在marked文件上执行grep命令进行查看文件代码\\
 & C-x C-h & 显示隐藏文件(默认配置了不显示)\\
\end{longtable}
\subsection{Tern}
\label{sec-3-5}


\begin{longtable}{l|l}
\caption{Tern快捷键}
\\
快捷键 & 描述\\
\hline
\endhead
\hline\multicolumn{2}{r}{Continued on next page} \\
\endfoot
\endlastfoot
M-. & 跳转到当前所在的参数或方法的定义位置\\
M-, & 返回刚在执行M-.的位置\\
C-c C-c & 重命名当前变量\\
C-c C-d & 找到当前变量的文档,再按就是打开它的文档中的URL\\
C-<tab> & 自动提示\\
\end{longtable}

\subsection{Wanderlust \# Removed package, using mutt instead}
\label{sec-3-6}
查看官方文档:\url{http://www.gohome.org/wl/doc/wl_toc.html};
\begin{longtable}{l|l}
\caption{快捷键}
\\
类别 & 键位 & 功能描述\\
\hline
\endhead
\hline\multicolumn{3}{r}{Continued on next page} \\
\endfoot
\endlastfoot
Summary & l & 打开/关闭左边的目录导航\\
 & f & 打开unread的summary\\
 & SPAC/RET & 查看邮件内容\\
 & n & 查看下一条邮件\\
 & p & 查看上一条邮件\\
 & S-n & 查看下一条未查看邮件\\
 & S-p & 查看上一条未查看邮件\\
 & S-s & 按字段进行邮件排序\\
 & j & 进入到详情页面或列表页面\\
\hline
Draft & w & 新建邮件\\
 & a & 回复邮件,只回复发邮件的人\\
 & C-u A & 回复所有\\
 & C-a & 回复全部,与上相同\\
 & C-x C-s & 保存\\
 & C-c C-x Tab & 添加附件\\
 & C-c C-c & 发送邮件\\
 & C-x C-k & 删除当前\\
 & C-c C-s & 发送并不删除draft\\
 & C-c C-o & 打开其它的draft如果有\\
\hline
Address Manager & C-c C-a & 进入地址管理\\
 & t & 添加To\\
 & c & 添加Cc\\
 & u & 取消添加\\
 & b & 添加Bcc\\
 & x & 添加to, cc, bcc, 并退出address manager\\
 & q & 退出地址管理\\
 & a & 添加entry\\
 & d & 删除entry\\
 & e & 修改entry\\
\hline
\end{longtable}

\subsection{Projectile \& helm}
\label{sec-3-7}
Project address: \url{https://github.com/bbatsov/projectile}
\begin{table}[htb]
\caption{绑定helm后的快捷键}
\centering
\begin{tabular}{l{2cm}|lp{2cm}r|lp{2cm}r|l{2cm}|l}
分类 & 快捷键 & 描述 & 掌握重点 & \\
\hline
基本查找 & C-c p h/C-c h & 打开helm-projectile,查看当前管理的所有项目及文件的全局搜索 & 常用 & \\
 & C-c p d & 查找项目中的文件夹 & 常用 & 需要在helm项目视图下执行\\
 & C-c p e & 打开近期打开的文件 & 常用 & \\
 & C-c p a & 打开当前名称相同的另一个后缀不相同的文件(js/css名称相同时用) & 常用 & \\
 & C-c p i & 刷新项目文件缓存 & 有时 & \\
 & C-c p z & 将当前文件添加到项目中 &  & \\
\hline
项目管理 & C-c p p & 当配置helm直接接管projectile后,可以直接用projectile项目切换快捷键 & 常用 & 以下命令是在项目视图下执行对应的Action\\
 & C-d & 使用Dired打开项目地址目录 & 常用 & 需要在helm项目视图下执行\\
 & M-g & 打开项目root目录 & 常用 & 需要在helm项目视图下执行\\
 & M-e & 在项目中打开Eshell &  & 需要在helm项目视图下执行\\
 & C-s & 使用grep命令 &  & 需要在helm项目视图下执行\\
 & C-u C-s & 使用grep进行递归查找 &  & 需要在helm项目视图下执行\\
 & C-c & 执行编译命令(可配置) &  & 需要在helm项目视图下执行\\
 & M-D & 删除项目 &  & \\
\hline
文件管理 & C-c p f & 在项目中查找文件 & 常用 & \\
 & M-SPC & 标记当前文件 &  & \\
 & M-a & 标记所有文件 &  & \\
 & C-c o & 在新窗口中打开文件 &  & \\
 & C-c C-o & 用新frame打开文件 &  & \\
 & C-c C-x & 使用外部程序打开文件 &  & \\
 & C-c r & 用root打开文件 &  & \\
 & M-R & 对文件进行重命名,通过M-SPC选择文件,通过M-R对文件进行重命名或移动操作 &  & \\
 & M-C & 拷贝文件 &  & \\
 & M-D & 删除文件 &  & \\
 & C-c p g & 重新匹配输入的命令,用于在未发现文件时的操作 &  & \\
\hline
缓冲管理 & C-c p b & 在项目中切换buffer &  & \\
\hline
项目搜索 & C-c p s g & 项目中搜索内容 & 常用 & \\
 & C-c p s a & 使用ack搜索内容 &  & \\
 & C-c p s s & 使用ag搜索内容 &  & \\
\hline
项目管理 &  &  &  & \\
\end{tabular}
\end{table}

\subsection{Projectile}
\label{sec-3-8}
\begin{longtable}{l|l|l}
\caption{Projectile快捷键收集}
\\
快捷键 & 描述\\
C-c p f & Display a list of all files in the project. With a prefix argument it will clear the cache first.\\
C-c p F & Display a list of all files in all known projects.\\
C-c p g & Display a list of all files at point in the project. With a prefix argument it will clear the cache first.\\
C-c p 4 f & Jump to a project's file using completion and show it in another window.\\
C-c p 4 g & Jump to a project's file based on context at point and show it in another window.\\
C-c p d & Display a list of all directories in the project. With a prefix argument it will clear the cache first.\\
C-c p 4 d & Switch to a project directory and show it in another window.\\
C-c p 4 a & Switch between files with the same name but different extensions in other window.\\
C-c p T & Display a list of all test files(specs, features, etc) in the project.\\
C-c p l & Display a list of all files in a directory (that's not necessarily a project)\\
C-c p s g & Run grep on the files in the project.\\
M-- C-c p s g & Run grep on projectile-grep-default-files in the project.\\
C-c p v & Run vc-dir on the root directory of the project.\\
C-c p b & Display a list of all project buffers currently open.\\
C-c p 4 b & Switch to a project buffer and show it in another window.\\
C-c p 4 C-o & Display a project buffer in another window without selecting it.\\
C-c p a & Switch between files with the same name but different extensions.\\
C-c p o & Runs multi-occur on all project buffers currently open.\\
C-c p r & Runs interactive query-replace on all files in the projects.\\
C-c p i & Invalidates the project cache (if existing).\\
C-c p R & Regenerates the projects TAGS file.\\
C-c p j & Find tag in project's TAGS file.\\
C-c p k & Kills all project buffers.\\
C-c p D & Opens the root of the project in dired.\\
C-c p e & Shows a list of recently visited project files.\\
C-c p s s & Runs ag on the project. Requires the presence of ag.el.\\
C-c p ! & Runs shell-command in the root directory of the project.\\
C-c p \& & Runs async-shell-command in the root directory of the project.\\
C-c p c & Runs a standard compilation command for your type of project.\\
C-c p P & Runs a standard test command for your type of project.\\
C-c p t & Toggle between an implementation file and its test file.\\
C-c p 4 t & Jump to implementation or test file in other window.\\
C-c p z & Adds the currently visited file to the cache.\\
C-c p p & Display a list of known projects you can switch to.\\
C-c p S & Save all project buffers.\\
C-c p m & Run the commander (an interface to run commands with a single key).\\
C-c p ESC & Switch to the most recently selected projectile buffer.\\
\end{longtable}

\subsection{SMIX}
\label{sec-3-9}
\begin{longtable}{l|l|l}
\caption{SMIX快捷键}
\\
类别 & 快捷键 & 描述\\
SMIX & M-x & 打开SMIX\\
\end{longtable}

\subsection{Tabbar}
\label{sec-3-10}
\begin{longtable}{l|l}
\caption{Tabbar切换快捷键}
\\
快捷键 & 描述\\
\hline
\endhead
\hline\multicolumn{2}{r}{Continued on next page} \\
\endfoot
\endlastfoot
C-c t & 打开Tabbar的模式,接下来可以使用下面的按键切换文件\\
C-c C-left/right & 切换tab文件\\
C-c C-up/down & 按group进行切换,启用了自动识别按projectile进行自动分组\\
\end{longtable}
\subsection{Magit}
\label{sec-3-11}
Magit是通过emacs操作git命令的工具
\begin{longtable}{l|l|l}
\caption{magit操作快捷键}
\\
类别 & 快捷键 & 描述\\
\hline
\endhead
\hline\multicolumn{3}{r}{Continued on next page} \\
\endfoot
\endlastfoot
基本 & M-x magit-status & 打开magit,查看修改记录\\
 & s & 进入到修改清单中,将此文件加入到staging\\
操作清单 & c & 进入magit操作菜单\\
 & c & 在操作清单中执行commit,输入commit信息\\
 & C-c C-c & 输入commit信息后,提交\\
 & P P & 推送到远程master, 输入用户名,密码即可提交\\
 & F F & 执行git pull\\
 & b b & 切换到其它的分支\\
\end{longtable}
\subsection{Latex}
\label{sec-3-12}
\begin{longtable}{l|l|l}
\caption{Latex快捷键}
\\
分类 & 快捷键 & 描述\\
\hline
\endhead
\hline\multicolumn{3}{r}{Continued on next page} \\
\endfoot
\endlastfoot
基本命令 & C-c C-e & 打开操作面板\\
 & l p & 导出\\
 & C-u C-c C-x C-l & 预览\\
\end{longtable}
\subsection{Sunrise Commander}
\label{sec-3-13}
\begin{longtable}{l|l|l}
\caption{Latex快捷键}
\\
分类 & 快捷键 & 描述\\
\hline
\endhead
\hline\multicolumn{3}{r}{Continued on next page} \\
\endfoot
\endlastfoot
基本命令 & C-c x & 打开sunrise窗口\\
 & C-c X & 打开sunrise-cd窗口\\
 & M-x customize-group RET sunrise RET & 查看sunrise的命令\\
 & C-c C-n & 按命名查找\\
 & C-c C-g & 按代码grep查找\\
 & C-c C-f & 查找\\
\end{longtable}

\subsection{Markdown}
\label{sec-3-14}
\begin{longtable}{l|l|l}
\caption{Markdown快捷键}
\\
分类 & 快捷键 & 描述\\
\hline
\endhead
\hline\multicolumn{3}{r}{Continued on next page} \\
\endfoot
\endlastfoot
编辑命令 & C-c C-t n & 插入 hash 样式的标题,其中 n 为 1\textasciitilde{}5,表示从第一级标题到第五级标题。\\
 & C-c C-t t & 插入 underline 样式的标题,这是一级。\\
 & C-c C-t s & 同上,这是二级。\\
 & C-c C-a l & 插入链接,格式为 [text](url)。\\
 & C-c C-i i & 插入图像,格式为 ![text](url)。\\
 & C-c C-s b & 插入引用内容。\\
 & C-c C-s c & 插入代码。\\
 & C-c C-p b & 加粗。\\
 & C-c C-p i & 斜体。\\
 & C-c - & 插入水平线。\\
\hline
大纲模式 & S-Tab & 在大纲视图、目录视图、及正常视图间切换\\
 &  & \\
\hline
预览 & C-c C-c m & 在当前缓冲运行 Markdown,并在另一个缓冲预览\\
 & C-c C-c p & 同上,但在浏览器中预览\\
\end{longtable}

\subsection{Graphviz-dot-mode}
\label{sec-3-15}
\begin{longtable}{l|l|l}
\caption{Graphviz-dot快捷键}
\\
分类 & 快捷键 & 描述\\
\hline
\endhead
\hline\multicolumn{3}{r}{Continued on next page} \\
\endfoot
\endlastfoot
编辑命令 & C-c c & compile\\
 & C-c p & viewing an generated image\\
\end{longtable}

\subsection{Read Code}
\label{sec-3-16}
\begin{itemize}
\item 生成tag文件:
\end{itemize}
\lstset{basicstyle=\color{foreground}\small\mono,keywordstyle=\color{function}\bfseries\small\mono,identifierstyle=\color{doc}\small\mono,commentstyle=\color{comment}\small\itshape,stringstyle=\color{string}\small,showstringspaces=false,numbers=left,numberstyle=\color{preprocess},stepnumber=1,backgroundcolor=\color{background},tabsize=4,captionpos=t,breaklines=true,breakatwhitespace=true,showspaces=false,columns=flexible,frame=single,frameround=tttt,framesep=0pt,framerule=8pt,rulecolor=\color{background},fillcolor=\color{white},rulesepcolor=\color{comdil},framexleftmargin=10mm,language=sh,label= ,caption= }
\begin{lstlisting}
find . -name "*.[chCHS]" | etags -
\end{lstlisting}

\begin{longtable}{l|l|l}
\caption{阅读源码快捷键}
\\
分类 & 快捷键 & 描述\\
\hline
\endhead
\hline\multicolumn{3}{r}{Continued on next page} \\
\endfoot
\endlastfoot
打开文件 & M-x visit-tag-table & 选择刚生成的TAGS文件\\
编辑命令 & M-. & 查找光标所指向的函数的定义\\
 & C-M-. & 输入函数名,查找其定义\\
 & M-* & 回退\\
 & C-u M-. & 查找标签的下一个定义\\
\end{longtable}
\subsection{Magit}
\label{sec-3-17}
\begin{longtable}{l|l|l}
\caption{git管理}
\\
分类 & 快捷键 & 描述\\
\hline
\endhead
\hline\multicolumn{3}{r}{Continued on next page} \\
\endfoot
\endlastfoot
打开 & C-x m & 打开magit-status\\
 & ? & 打开帮助\\
 & s & 提交文件到stage\\
 & c & 提交到本地master\\
 & P & 提交到远程分支\\
 & F & 执行一次git pull\\
\end{longtable}
\section{Packages Management}
\label{sec-4}
\subsubsection{Package List}
\label{sec-4-0-1}
\begin{center}
\begin{tabular}{ll}
Package name & Markdown\\
\hline
Projectile & \\
helm-projectile & \\
project-codesearch & \\
helm & \\
wanderlust & \\
anything & \\
dired+ & \\
dired-details & \\
dired-details+ & \\
dired-sort & \\
expand-region & \\
js2-refactor & \\
jump-char & \\
multifiles & \\
multiple-cursors & \\
paredit & \\
perspective & \\
skewer & \\
smart-forward & \\
smex & \\
yasnippet & \\
zencoding-mode & \\
codesearch & \\
boxquote & \\
magit & \\
simple-httpd & \\
height-symbol & \\
ido-completing-read+ & \\
ox-twbs & \\
tern & \url{http://ternjs.net}\\
tidy & 需要安装Tidyhtml工具,并配置\textasciitilde{}/.tidyrc文件\\
Auctex & \url{http://www.gnu.org/software/auctex/}\\
Plantuml & \url{https://github.com/wildsoul/plantuml-mode}\\
 & \\
\end{tabular}
\end{center}
\subsubsection{需要重点学习的包}
\label{sec-4-0-2}
\begin{enumerate}
\item Projectile
\label{sec-4-0-2-1}
\item Helm
\label{sec-4-0-2-2}
\item Smex
\label{sec-4-0-2-3}
\item Wanderlust  \# has been removed
\label{sec-4-0-2-4}
\item Org-mode
\label{sec-4-0-2-5}
\item Markdown
\label{sec-4-0-2-6}
\item Ido
\label{sec-4-0-2-7}
\item zencoding-mode
\label{sec-4-0-2-8}
\item yasnippet
\label{sec-4-0-2-9}
\item dired
\label{sec-4-0-2-10}
\item anything
\label{sec-4-0-2-11}
\item wgrep
\label{sec-4-0-2-12}
\item skewer
\label{sec-4-0-2-13}
\item tern/tern-server
\label{sec-4-0-2-14}
\item ielm
\label{sec-4-0-2-15}
\item helm-css-sass
\label{sec-4-0-2-16}
\item helm-swoop
\label{sec-4-0-2-17}
\item tabbar-ruler
\label{sec-4-0-2-18}
\item Tramp
\label{sec-4-0-2-19}
\item Magit
\label{sec-4-0-2-20}
\item Tidy
\label{sec-4-0-2-21}
\item Speedbar
\label{sec-4-0-2-22}
\item Latex
\label{sec-4-0-2-23}
\begin{itemize}
\item 安装texlive: sudo pacman -S texlive-bin texlive-core texlive-fontsextra texlive-formatextra texlive-langchinese texlive-langcjk texlive-langextra texlive-picture
\item 安装字体:sudo pacman -S adobe-source-han-sans-otc-fonts wqy-microhei
\item 网上下载sim字库:宋体(simsun)、黑体(simhei)、仿宋体(simfang)、楷体(simkai)
\item 确保在org文件上包含了第一节里的header中latex的字体定义
\end{itemize}
\end{enumerate}
\section{Supplement}
\label{sec-5}
\subsection{Org-mode Key-bindings from official}
\label{sec-5-1}
From : \url{http://orgmode.org/orgcard.txt}

\texttt{==============================================================================}
Org-Mode Reference Card (for version 7.8.11)
\texttt{==============================================================================}



\texttt{==============================================================================}
Getting Started
\texttt{==============================================================================}
To read the on-line documentation try             M-x org-info

\texttt{==============================================================================}
Visibility Cycling
\texttt{==============================================================================}

rotate current subtree between states             TAB
rotate entire buffer between states               S-TAB
restore property-dependent startup visibility     C-u C-u TAB
show the whole file, including drawers            C-u C-u C-u TAB
reveal context around point                       C-c C-r

\texttt{==============================================================================}
Motion
\texttt{==============================================================================}

next/previous heading                             C-c C-n/p
next/previous heading, same level                 C-c C-f/b
backward to higher level heading                  C-c C-u
jump to another place in document                 C-c C-j
previous/next plain list item                     S-UP/DOWN\notetwo

\texttt{==============================================================================}
Structure Editing
\texttt{==============================================================================}

insert new heading/item at current level          M-RET
insert new heading after subtree                  C-RET
insert new TODO entry/checkbox item               M-S-RET
insert TODO entry/ckbx after subtree              C-S-RET
turn (head)line into item, cycle item type        C-c -
turn item/line into headline                      C-c *
promote/demote heading                            M-LEFT/RIGHT
promote/demote current subtree                    M-S-LEFT/RIGHT
move subtree/list item up/down                    M-S-UP/DOWN
sort subtree/region/plain-list                    C-c \^{}
clone a subtree                                   C-c C-x c
copy visible text                                 C-c C-x v
kill/copy subtree                                 C-c C-x C-w/M-w
yank subtree                                      C-c C-x C-y or C-y
narrow buffer to subtree / widen                  C-x n s/w

\texttt{==============================================================================}
Capture - Refile - Archiving
\texttt{==============================================================================}
capture a new item (C-u C-u = goto last)          C-c c \noteone
refile subtree (C-u C-u = goto last)              C-c C-w
archive subtree using the default command         C-c C-x C-a
move subtree to archive file                      C-c C-x C-s
toggle ARCHIVE tag / to ARCHIVE sibling           C-c C-x a/A
force cycling of an ARCHIVEd tree                 C-TAB

\texttt{==============================================================================}
Filtering and Sparse Trees
\texttt{==============================================================================}

construct a sparse tree by various criteria       C-c /
view TODO's in sparse tree                        C-c / t/T
global TODO list in agenda mode                   C-c a t \noteone
time sorted view of current org file              C-c a L

\texttt{==============================================================================}
Tables
\texttt{==============================================================================}

\rule{\linewidth}{0.5pt}
Creating a table

\rule{\linewidth}{0.5pt}

just start typing, e.g.                           |Name|Phone|Age RET |- TAB
convert region to table                           C-c |
\ldots{} separator at least 3 spaces                   C-3 C-c |

\rule{\linewidth}{0.5pt}
Commands available inside tables

\rule{\linewidth}{0.5pt}

The following commands work when the cursor is inside a table.
Outside of tables, the same keys may have other functionality.

\rule{\linewidth}{0.5pt}
Re-aligning and field motion

\rule{\linewidth}{0.5pt}

re-align the table without moving the cursor      C-c C-c
re-align the table, move to next field            TAB
move to previous field                            S-TAB
re-align the table, move to next row              RET
move to beginning/end of field                    M-a/e

\rule{\linewidth}{0.5pt}
Row and column editing

\rule{\linewidth}{0.5pt}

move the current column left                      M-LEFT/RIGHT
kill the current column                           M-S-LEFT
insert new column to left of cursor position      M-S-RIGHT

move the current row up/down                      M-UP/DOWN
kill the current row or horizontal line           M-S-UP
insert new row above the current row              M-S-DOWN
insert hline below (C-u : above) current row      C-c -
insert hline and move to line below it            C-c RET
sort lines in region                              C-c \^{}

\rule{\linewidth}{0.5pt}
Regions

\rule{\linewidth}{0.5pt}

cut/copy/paste rectangular region                 C-c C-x C-w/M-w/C-y
fill paragraph across selected cells              C-c C-q

\rule{\linewidth}{0.5pt}
Miscellaneous

\rule{\linewidth}{0.5pt}

to limit column width to N characters, use        \ldots{}| <N> |\ldots{}
edit the current field in a separate window       C-c `
make current field fully visible                  C-u TAB
export as tab-separated file                      M-x org-table-export
import tab-separated file                         M-x org-table-import
sum numbers in current column/rectangle           C-c +

\rule{\linewidth}{0.5pt}
Tables created with the table.el package

\rule{\linewidth}{0.5pt}

insert a new table.el table                       C-c \textasciitilde{}
recognize existing table.el table                 C-c C-c
convert table (Org-mode <-> table.el)             C-c \textasciitilde{}

\rule{\linewidth}{0.5pt}
Spreadsheet

\rule{\linewidth}{0.5pt}

Formulas typed in field are executed by TAB,
RET and C-c C-c.  = introduces a column
formula, := a field formula.

Example: Add Col1 and Col2                        |=\$1+\$2      |
\ldots{} with printf format specification              |=\$1+\$2;\%.2f|
\ldots{} with constants from constants.el              |=\$1/\$c/\$cm |
sum from 2nd to 3rd hline                         |:=vsum(@II..@III)|
apply current column formula                      | = |

set and eval column formula                       C-c =
set and eval field formula                        C-u C-c =
re-apply all stored equations to current line     C-c *
re-apply all stored equations to entire table     C-u C-c *
iterate table to stability                        C-u C-u C-c *
rotate calculation mark through \# * ! \^{} \_ \$       C-\#
show line, column, formula reference              C-c ?
toggle grid / debugger                            C-c \}/\{

\rule{\linewidth}{0.5pt}
Formula Editor

\rule{\linewidth}{0.5pt}

edit formulas in separate buffer                  C-c '
exit and install new formulas                     C-c C-c
exit, install, and apply new formulas             C-u C-c C-c
abort                                             C-c C-q
toggle reference style                            C-c C-r
pretty-print Lisp formula                         TAB
complete Lisp symbol                              M-TAB
shift reference point                             S-cursor
shift test line for column references             M-up/down
scroll the window showing the table               M-S-up/down
toggle table coordinate grid                      C-c \}

\texttt{==============================================================================}
Links
\texttt{==============================================================================}

globally store link to the current location       C-c l \noteone
insert a link (TAB completes stored links)        C-c C-l
insert file link with file name completion        C-u C-c C-l
edit (also hidden part of) link at point          C-c C-l

open file links in emacs                          C-c C-o
\ldots{}force open in emacs/other window               C-u C-c C-o
open link at point                                mouse-1/2
\ldots{}force open in emacs/other window               mouse-3
record a position in mark ring                    C-c \%
jump back to last followed link(s)                C-c \&
find next link                                    C-c C-x C-n
find previous link                                C-c C-x C-p
edit code snippet of file at point                C-c '
toggle inline display of linked images            C-c C-x C-v

\texttt{==============================================================================}
Working with Code (Babel)
\texttt{==============================================================================}

execute code block at point                       C-c C-c
open results of code block at point               C-c C-o
check code block at point for errors              C-c C-v c
insert a header argument with completion          C-c C-v j
view expanded body of code block at point         C-c C-v v
view information about code block at point        C-c C-v I
go to named code block                            C-c C-v g
go to named result                                C-c C-v r
go to the head of the current code block          C-c C-v u
go to the next code block                         C-c C-v n
go to the previous code block                     C-c C-v p
demarcate a code block                            C-c C-v d
execute the next key sequence in the code edit bu C-c C-v x
execute all code blocks in current buffer         C-c C-v b
execute all code blocks in current subtree        C-c C-v s
tangle code blocks in current file                C-c C-v t
tangle code blocks in supplied file               C-c C-v f
ingest all code blocks in supplied file into the  C-c C-v i
switch to the session of the current code block   C-c C-v z
load the current code block into a session        C-c C-v l
view sha1 hash of the current code block          C-c C-v a

\texttt{==============================================================================}
Completion
\texttt{==============================================================================}

In-buffer completion completes TODO keywords at headline start, \TeX{}
macros after `$\backslash$', option keywords after `\#-', TAGS
after  `:', and dictionary words elsewhere.

complete word at point                            M-TAB




\texttt{==============================================================================}
TODO Items and Checkboxes
\texttt{==============================================================================}

rotate the state of the current item              C-c C-t
select next/previous state                        S-LEFT/RIGHT
select next/previous set                          C-S-LEFT/RIGHT
toggle ORDERED property                           C-c C-x o
view TODO items in a sparse tree                  C-c C-v
view 3rd TODO keyword's sparse tree               C-3 C-c C-v

set the priority of the current item              C-c , [ABC]
remove priority cookie from current item          C-c , SPC
raise/lower priority of current item              S-UP/DOWN\notetwo

insert new checkbox item in plain list            M-S-RET
toggle checkbox(es) in region/entry/at point      C-c C-x C-b
toggle checkbox at point                          C-c C-c
update checkbox statistics (C-u : whole file)     C-c \#

\texttt{==============================================================================}
Tags
\texttt{==============================================================================}

set tags for current heading                      C-c C-q
realign tags in all headings                      C-u C-c C-q
create sparse tree with matching tags             C-c \\
globally (agenda) match tags at cursor            C-c C-o

\texttt{==============================================================================}
Properties and Column View
\texttt{==============================================================================}

set property/effort                               C-c C-x p/e
special commands in property lines                C-c C-c
next/previous allowed value                       S-left/right
turn on column view                               C-c C-x C-c
capture columns view in dynamic block             C-c C-x i

quit column view                                  q
show full value                                   v
edit value                                        e
next/previous allowed value                       n/p or S-left/right
edit allowed values list                          a
make column wider/narrower                        > / <
move column left/right                            M-left/right
add new column                                    M-S-right
Delete current column                             M-S-left


\texttt{==============================================================================}
Timestamps
\texttt{==============================================================================}

prompt for date and insert timestamp              C-c .
like C-c . but insert date and time format        C-u C-c .
like C-c . but make stamp inactive                C-c !
insert DEADLINE timestamp                         C-c C-d
insert SCHEDULED timestamp                        C-c C-s
create sparse tree with all deadlines due         C-c / d
the time between 2 dates in a time range          C-c C-y
change timestamp at cursor ±1 day                S-RIGHT/LEFT\notetwo
change year/month/day at cursor by ±1            S-UP/DOWN\notetwo
access the calendar for the current date          C-c >
insert timestamp matching date in calendar        C-c <
access agenda for current date                    C-c C-o
select date while prompted                        mouse-1/RET
toggle custom format display for dates/times      C-c C-x C-t


\rule{\linewidth}{0.5pt}
Clocking time

\rule{\linewidth}{0.5pt}

start clock on current item                       C-c C-x C-i
stop/cancel clock on current item                 C-c C-x C-o/x
display total subtree times                       C-c C-x C-d
remove displayed times                            C-c C-c
insert/update table with clock report             C-c C-x C-r

\texttt{==============================================================================}
Agenda Views
\texttt{==============================================================================}

add/move current file to front of agenda          C-c [
remove current file from your agenda              C-c ]
cycle through agenda file list                    C-'
set/remove restriction lock                       C-c C-x </>

compile agenda for the current week               C-c a a \noteone
compile global TODO list                          C-c a t \noteone
compile TODO list for specific keyword            C-c a T \noteone
match tags, TODO kwds, properties                 C-c a m \noteone
match only in TODO entries                        C-c a M \noteone
find stuck projects                               C-c a \# \noteone
show timeline of current org file                 C-c a L \noteone
configure custom commands                         C-c a C \noteone
agenda for date at cursor                         C-c C-o

\rule{\linewidth}{0.5pt}
Commands available in an agenda buffer

\rule{\linewidth}{0.5pt}

\rule{\linewidth}{0.5pt}
View Org file

\rule{\linewidth}{0.5pt}

show original location of item                    SPC/mouse-3
show and recenter window                          L
goto original location in other window            TAB/mouse-2
goto original location, delete other windows      RET
show subtree in indirect buffer, ded.$\backslash$ frame      C-c C-x b
toggle follow-mode                                F

\rule{\linewidth}{0.5pt}
Change display

\rule{\linewidth}{0.5pt}

delete other windows                              o
view mode dispatcher                              v
switch to day/week/month/year/def view            d w vm vy vSP
toggle diary entries / time grid / habits         D / G / K
toggle entry text / clock report                  E / R
toggle display of logbook entries                 l / v l/L/c
toggle inclusion of archived trees/files          v a/A
refresh agenda buffer with any changes            r / g
filter with respect to a tag                      /
save all org-mode buffers                         s
display next/previous day,week,\ldots{}                f / b
goto today / some date (prompt)                   . / j

\rule{\linewidth}{0.5pt}
Remote editing

\rule{\linewidth}{0.5pt}

digit argument                                    0-9
change state of current TODO item                 t
kill item and source                              C-k
archive default                                   \$ / a
refile the subtree                                C-c C-w
set/show tags of current headline                 : / T
set effort property (prefix=nth)                  e
set / compute priority of current item            , / P
raise/lower priority of current item              S-UP/DOWN\notetwo
run an attachment command                         C-c C-a
schedule/set deadline for this item               C-c C-s/d
change timestamp one day earlier/later            S-LEFT/RIGHT\notetwo
change timestamp to today                         >
insert new entry into diary                       i
start/stop/cancel the clock on current item       I / O / X
jump to running clock entry                       J
mark / unmark / execute bulk action               m / u / B

\rule{\linewidth}{0.5pt}
Misc

\rule{\linewidth}{0.5pt}

follow one or offer all links in current entry    C-c C-o

\rule{\linewidth}{0.5pt}
Calendar commands

\rule{\linewidth}{0.5pt}

find agenda cursor date in calendar               c
compute agenda for calendar cursor date           c
show phases of the moon                           M
show sunrise/sunset times                         S
show holidays                                     H
convert date to other calendars                   C

\rule{\linewidth}{0.5pt}
Quit and Exit

\rule{\linewidth}{0.5pt}

quit agenda, remove agenda buffer                 q
exit agenda, remove all agenda buffers            x

\texttt{==============================================================================}
\LaTeX{} and cdlatex-mode
\texttt{==============================================================================}

preview \LaTeX{} fragment                            C-c C-x C-l
expand abbreviation (cdlatex-mode)                TAB
insert/modify math symbol (cdlatex-mode)          ` / '
insert citation using RefTeX                      C-c C-x [

\texttt{==============================================================================}
Exporting and Publishing
\texttt{==============================================================================}

Exporting creates files with extensions .txt and .html
in the current directory.  Publishing puts the resulting file into
some other place.

export/publish dispatcher                         C-c C-e

export visible part only                          C-c C-e v
insert template of export options                 C-c C-e t
toggle fixed width for entry or region            C-c :
toggle pretty display of scripts, entities        C-c C-x \{\tt\char`\}

\rule{\linewidth}{0.5pt}
Comments: Text not being exported

\rule{\linewidth}{0.5pt}

Lines starting with \# and subtrees starting with COMMENT are
never exported.

toggle COMMENT keyword on entry                   C-c ;

\texttt{==============================================================================}
Dynamic Blocks
\texttt{==============================================================================}

update dynamic block at point                     C-c C-x C-u
update all dynamic blocks                         C-u C-c C-x C-u

\texttt{==============================================================================}
Notes
\texttt{==============================================================================}
Emacs 24.5.1 (Org mode 8.2.10)
\end{document}